% Options for packages loaded elsewhere
\PassOptionsToPackage{unicode}{hyperref}
\PassOptionsToPackage{hyphens}{url}
\PassOptionsToPackage{dvipsnames,svgnames,x11names}{xcolor}
%
\documentclass[
]{article}
\usepackage{amsmath,amssymb}
\usepackage{iftex}
\ifPDFTeX
  \usepackage[T1]{fontenc}
  \usepackage[utf8]{inputenc}
  \usepackage{textcomp} % provide euro and other symbols
\else % if luatex or xetex
  \usepackage{unicode-math} % this also loads fontspec
  \defaultfontfeatures{Scale=MatchLowercase}
  \defaultfontfeatures[\rmfamily]{Ligatures=TeX,Scale=1}
\fi
\usepackage{lmodern}
\ifPDFTeX\else
  % xetex/luatex font selection
\fi
% Use upquote if available, for straight quotes in verbatim environments
\IfFileExists{upquote.sty}{\usepackage{upquote}}{}
\IfFileExists{microtype.sty}{% use microtype if available
  \usepackage[]{microtype}
  \UseMicrotypeSet[protrusion]{basicmath} % disable protrusion for tt fonts
}{}
\makeatletter
\@ifundefined{KOMAClassName}{% if non-KOMA class
  \IfFileExists{parskip.sty}{%
    \usepackage{parskip}
  }{% else
    \setlength{\parindent}{0pt}
    \setlength{\parskip}{6pt plus 2pt minus 1pt}}
}{% if KOMA class
  \KOMAoptions{parskip=half}}
\makeatother
\usepackage{xcolor}
\usepackage[margin=1in]{geometry}
\usepackage{color}
\usepackage{fancyvrb}
\newcommand{\VerbBar}{|}
\newcommand{\VERB}{\Verb[commandchars=\\\{\}]}
\DefineVerbatimEnvironment{Highlighting}{Verbatim}{commandchars=\\\{\}}
% Add ',fontsize=\small' for more characters per line
\usepackage{framed}
\definecolor{shadecolor}{RGB}{248,248,248}
\newenvironment{Shaded}{\begin{snugshade}}{\end{snugshade}}
\newcommand{\AlertTok}[1]{\textcolor[rgb]{0.94,0.16,0.16}{#1}}
\newcommand{\AnnotationTok}[1]{\textcolor[rgb]{0.56,0.35,0.01}{\textbf{\textit{#1}}}}
\newcommand{\AttributeTok}[1]{\textcolor[rgb]{0.13,0.29,0.53}{#1}}
\newcommand{\BaseNTok}[1]{\textcolor[rgb]{0.00,0.00,0.81}{#1}}
\newcommand{\BuiltInTok}[1]{#1}
\newcommand{\CharTok}[1]{\textcolor[rgb]{0.31,0.60,0.02}{#1}}
\newcommand{\CommentTok}[1]{\textcolor[rgb]{0.56,0.35,0.01}{\textit{#1}}}
\newcommand{\CommentVarTok}[1]{\textcolor[rgb]{0.56,0.35,0.01}{\textbf{\textit{#1}}}}
\newcommand{\ConstantTok}[1]{\textcolor[rgb]{0.56,0.35,0.01}{#1}}
\newcommand{\ControlFlowTok}[1]{\textcolor[rgb]{0.13,0.29,0.53}{\textbf{#1}}}
\newcommand{\DataTypeTok}[1]{\textcolor[rgb]{0.13,0.29,0.53}{#1}}
\newcommand{\DecValTok}[1]{\textcolor[rgb]{0.00,0.00,0.81}{#1}}
\newcommand{\DocumentationTok}[1]{\textcolor[rgb]{0.56,0.35,0.01}{\textbf{\textit{#1}}}}
\newcommand{\ErrorTok}[1]{\textcolor[rgb]{0.64,0.00,0.00}{\textbf{#1}}}
\newcommand{\ExtensionTok}[1]{#1}
\newcommand{\FloatTok}[1]{\textcolor[rgb]{0.00,0.00,0.81}{#1}}
\newcommand{\FunctionTok}[1]{\textcolor[rgb]{0.13,0.29,0.53}{\textbf{#1}}}
\newcommand{\ImportTok}[1]{#1}
\newcommand{\InformationTok}[1]{\textcolor[rgb]{0.56,0.35,0.01}{\textbf{\textit{#1}}}}
\newcommand{\KeywordTok}[1]{\textcolor[rgb]{0.13,0.29,0.53}{\textbf{#1}}}
\newcommand{\NormalTok}[1]{#1}
\newcommand{\OperatorTok}[1]{\textcolor[rgb]{0.81,0.36,0.00}{\textbf{#1}}}
\newcommand{\OtherTok}[1]{\textcolor[rgb]{0.56,0.35,0.01}{#1}}
\newcommand{\PreprocessorTok}[1]{\textcolor[rgb]{0.56,0.35,0.01}{\textit{#1}}}
\newcommand{\RegionMarkerTok}[1]{#1}
\newcommand{\SpecialCharTok}[1]{\textcolor[rgb]{0.81,0.36,0.00}{\textbf{#1}}}
\newcommand{\SpecialStringTok}[1]{\textcolor[rgb]{0.31,0.60,0.02}{#1}}
\newcommand{\StringTok}[1]{\textcolor[rgb]{0.31,0.60,0.02}{#1}}
\newcommand{\VariableTok}[1]{\textcolor[rgb]{0.00,0.00,0.00}{#1}}
\newcommand{\VerbatimStringTok}[1]{\textcolor[rgb]{0.31,0.60,0.02}{#1}}
\newcommand{\WarningTok}[1]{\textcolor[rgb]{0.56,0.35,0.01}{\textbf{\textit{#1}}}}
\usepackage{graphicx}
\makeatletter
\def\maxwidth{\ifdim\Gin@nat@width>\linewidth\linewidth\else\Gin@nat@width\fi}
\def\maxheight{\ifdim\Gin@nat@height>\textheight\textheight\else\Gin@nat@height\fi}
\makeatother
% Scale images if necessary, so that they will not overflow the page
% margins by default, and it is still possible to overwrite the defaults
% using explicit options in \includegraphics[width, height, ...]{}
\setkeys{Gin}{width=\maxwidth,height=\maxheight,keepaspectratio}
% Set default figure placement to htbp
\makeatletter
\def\fps@figure{htbp}
\makeatother
\setlength{\emergencystretch}{3em} % prevent overfull lines
\providecommand{\tightlist}{%
  \setlength{\itemsep}{0pt}\setlength{\parskip}{0pt}}
\setcounter{secnumdepth}{-\maxdimen} % remove section numbering
\ifLuaTeX
  \usepackage{selnolig}  % disable illegal ligatures
\fi
\usepackage{bookmark}
\IfFileExists{xurl.sty}{\usepackage{xurl}}{} % add URL line breaks if available
\urlstyle{same}
\hypersetup{
  pdftitle={Assignment 2},
  pdfauthor={Jiacong Zheng(S3913565)},
  colorlinks=true,
  linkcolor={Maroon},
  filecolor={Maroon},
  citecolor={Blue},
  urlcolor={blue},
  pdfcreator={LaTeX via pandoc}}

\title{Assignment 2}
\usepackage{etoolbox}
\makeatletter
\providecommand{\subtitle}[1]{% add subtitle to \maketitle
  \apptocmd{\@title}{\par {\large #1 \par}}{}{}
}
\makeatother
\subtitle{Deconstruct, Reconstruct Web Report}
\author{Jiacong Zheng(S3913565)}
\date{}

\begin{document}
\maketitle

\subsubsection{Assessment declaration
checklist}\label{assessment-declaration-checklist}

Please carefully read the statements below and check each box if you
agree with the declaration. If you do not check all boxes, your
assignment will not be marked. If you make a false declaration on any of
these points, you may be investigated for academic misconduct. Students
found to have breached academic integrity may receive official warnings
and/or serious academic penalties. Please read more about academic
integrity
\href{https://www.rmit.edu.au/students/student-essentials/assessment-and-exams/academic-integrity}{here}.
If you are unsure about any of these points or feel your assessment
might breach academic integrity, please contact your course coordinator
for support. It is important that you DO NOT submit any assessment until
you can complete the declaration truthfully.

\textbf{By checking the boxes below, I declare the following:}

\begin{itemize}
\item
  I have not impersonated, or allowed myself to be impersonated by, any
  person for the purposes of this assessment
\item
  This assessment is my original work and no part of it has been copied
  from any other source except where due acknowledgement is made. Due
  acknowledgement means the following:

  \begin{itemize}
  \tightlist
  \item
    The source is correctly referenced in a reference list
  \item
    The work has been paraphrased or directly quoted
  \item
    A citation to the original work's reference has been included where
    the copied work appears in the assessment.
  \end{itemize}
\item
  No part of this assessment has been written for me by any other person
  except where such collaboration has been authorised by the
  lecturer/teacher concerned.
\item
  I have not used generative ``AI'' tools for the purposes of this
  assessment.
\item
  Where this work is being submitted for individual assessment, I
  declare that it is my original work and that no part has been
  contributed by, produced by or in conjunction with another student.
\item
  I give permission for my assessment response to be reproduced,
  communicated, compared and archived for the purposes of detecting
  plagiarism.
\item
  I give permission for a copy of my assessment to be retained by the
  university for review and comparison, including review by external
  examiners.
\end{itemize}

\textbf{I understand that:}

\begin{itemize}
\item
  Plagiarism is the presentation of the work, idea or creation of
  another person or machine as though it is your own. It is a form of
  cheating and is a very serious academic offence that may lead to
  exclusion from the University. Plagiarised material can be drawn from,
  and presented in, written, graphic and visual form, including
  electronic data and oral presentations. Plagiarism occurs when the
  origin of the material used is not appropriately cited.
\item
  Plagiarism includes the act of assisting or allowing another person to
  plagiarise or to copy my work.
\end{itemize}

\textbf{I agree and acknowledge that:}

\begin{itemize}
\item
  I have read and understood the Declaration and Statement of Authorship
  above.
\item
  If I do not agree to the Declaration and Statement of Authorship in
  this context and all boxes are not checked, the assessment outcome is
  not valid for assessment purposes and will not be included in my final
  result for this course.
\end{itemize}

\subsection{Deconstruct}\label{deconstruct}

\subsubsection{Original}\label{original}

The original data visualisation selected for the assignment was as
follows:

\emph{Source: Zena Chamas(2025).}

\subsubsection{Objective and Audience}\label{objective-and-audience}

The objective and audience of the original data visualisation chosen can
be summarised as follows:

\textbf{Objective} To show that more Australians are voting early than
in previous elections.

\textbf{Audience} People who concern about the Federal Election voting
this year. They could be the citizens and the politicians.

\subsubsection{Improvements}\label{improvements}

The original data visualisation chosen could be improved in the three
following ways:

\begin{itemize}
\tightlist
\item
  They included the `TOTAL' into the pie chart which has no sense having
  it, it wasted half of the pie charts. So I would improve it by
  removing `TOTAL' in the chart.
\item
  Pie chart is bad for comparison of the votings between states. So I
  would change it to bar chart.
\item
  The audiences would find hard to compare the votings between 2022 and
  2025 among the 2 pie charts. So I would improve it by making them to
  one single side-by-side bar chart
\end{itemize}

\subsection{Reconstruct}\label{reconstruct}

\subsubsection{Code}\label{code}

The following code was used to improve the original.

\begin{Shaded}
\begin{Highlighting}[]
\FunctionTok{library}\NormalTok{(ggplot2)}
\FunctionTok{library}\NormalTok{(tidyr) }

\NormalTok{raw\_data }\OtherTok{\textless{}{-}} \FunctionTok{read.csv}\NormalTok{(}\StringTok{"data.csv"}\NormalTok{)}
\CommentTok{\# Remove Total }
\NormalTok{filtered\_data }\OtherTok{\textless{}{-}}\NormalTok{ raw\_data[raw\_data}\SpecialCharTok{$}\NormalTok{STATE }\SpecialCharTok{!=} \StringTok{"TOTAL"}\NormalTok{, ]}

\NormalTok{data\_long }\OtherTok{\textless{}{-}} \FunctionTok{pivot\_longer}\NormalTok{(filtered\_data, }\AttributeTok{cols =} \SpecialCharTok{{-}}\NormalTok{STATE, }\AttributeTok{names\_to =} \StringTok{"YEAR"}\NormalTok{, }\AttributeTok{values\_to =} \StringTok{"VOTES"}\NormalTok{)}


\NormalTok{p }\OtherTok{\textless{}{-}} \FunctionTok{ggplot}\NormalTok{(data\_long, }\AttributeTok{mapping =} \FunctionTok{aes}\NormalTok{(}\AttributeTok{x=}\NormalTok{ STATE ,}\AttributeTok{y=}\NormalTok{VOTES, }\AttributeTok{fill=}\NormalTok{YEAR))}

\NormalTok{p }\OtherTok{\textless{}{-}}\NormalTok{ p }\SpecialCharTok{+} \FunctionTok{geom\_bar}\NormalTok{(}\AttributeTok{stat =} \StringTok{"identity"}\NormalTok{, }\AttributeTok{position =} \StringTok{"dodge"}\NormalTok{) }\SpecialCharTok{+} 
  \FunctionTok{labs}\NormalTok{(}\AttributeTok{title =} \StringTok{\textquotesingle{}Pre{-}poll voting figures of Federal election }\SpecialCharTok{\textbackslash{}n}\StringTok{ for the same time at the 2022 and 2025\textquotesingle{}}\NormalTok{) }\SpecialCharTok{+} 
  \FunctionTok{theme\_bw}\NormalTok{() }\SpecialCharTok{+}
  \FunctionTok{theme}\NormalTok{(}\AttributeTok{plot.title =} \FunctionTok{element\_text}\NormalTok{(}
    \AttributeTok{face =} \StringTok{"bold"}\NormalTok{,}
    \AttributeTok{size =} \DecValTok{16}\NormalTok{,}
    \AttributeTok{hjust =} \FloatTok{0.5}
\NormalTok{  ))}
\end{Highlighting}
\end{Shaded}

\subsubsection{Reconstruction}\label{reconstruction}

The following plot improves the original data visualisation in the three
ways previously explained.

\begin{center}\includegraphics{Assignment_2_RMarkdown_Template_files/figure-latex/unnamed-chunk-2-1} \end{center}

\subsection{References}\label{references}

The reference to the original data visualisation choose, the data
source(s) used for the reconstruction and any other sources used for
this assignment are as follows:

\begin{itemize}
\tightlist
\item
  Zena Chamas(2025). \emph{More Australians are voting early than in
  previous elections. Some parties have taken advantage}. Retrieved May
  03, 2025, from ABC News website:
  \url{https://www.abc.net.au/news/2025-04-30/how-many-people-have-voted-already-prepoll-numbers-fe2025/105227960}
\end{itemize}

\end{document}
